\section{Conclusions}

Using automated methods to find bugs is an important part of modern
compiler development.  For a variety of reasons, the use of
off-the-shelf fuzzers, especially AFL, is the most widely used such
approach.  Mutation-based fuzzers, however, were originally used to
test binary formats, and their heritage limits their effectiveness for
fuzzing compilers.  Most solutions to this problem either place significant
burden on compiler developers, or are ineffective, or both.  We show
that using ideas from mutation testing it is possible to significantly
improve AFL-based fuzzing of compilers without forcing developers to
provide grammars or a dictionary.  One of our two configurations
performed best for all compilers we investigated, in experiments,
sometimes dramatically so (finding more than twice as many bugs on
average as the best version of un-augmented AFL).  Using our approach
we reported more than 100 previously unknown bugs, that have been
fixed as a result, in important
compiler projects.

\section*{Acknowledgements}
\begin{sloppypar}
  A portion of this work was
  supported by the National Science Foundation under CCF-2129446.  The
authors would also like to thank our anonymous reviewers and, finally, the
developers of the various compilers we tested, who confirmed and fixed
the bugs we found, and in some cases actively changed the compiler or
its infrastructure to support our efforts.
\end{sloppypar}