%%
%% This is file `sample-sigconf.tex',
%% generated with the docstrip utility.
%%
%% The original source files were:
%%
%% samples.dtx  (with options: `sigconf')
%%
%% IMPORTANT NOTICE:
%%
%% For the copyright see the source file.
%%
%% Any modified versions of this file must be renamed
%% with new filenames distinct from sample-sigconf.tex.
%%
%% For distribution of the original source see the terms
%% for copying and modification in the file samples.dtx.
%%
%% This generated file may be distributed as long as the
%% original source files, as listed above, are part of the
%% same distribution. (The sources need not necessarily be
%% in the same archive or directory.)
%%
%% The first command in your LaTeX source must be the \documentclass command.
\documentclass[sigplan,review]{acmart}

\usepackage{code}
\usepackage{graphicx}
\usepackage{balance}
\usepackage{multirow}
\usepackage{multicol}
\usepackage{color, colortbl}
\usepackage{wasysym}
\usepackage{pifont}
\usepackage{listings}
\usepackage{subcaption}

\usepackage[most]{tcolorbox}
\tcbuselibrary{skins,breakable,listings}


%%% The following is specific to Onward! '21 and the paper
%%% 'Let a Thousand Flowers Bloom: On the Uses of Diversity in Software Testing'
%%% by Alex Groce.
%%%
\setcopyright{acmcopyright}
\acmPrice{15.00}
\acmDOI{10.1145/3486607.3486772}
\acmYear{2022}
\copyrightyear{2022}
\acmSubmissionID{onward21essays-id2-p}
\acmISBN{978-1-4503-9110-8/21/10}
\acmConference[foo]{bar}
\acmBooktitle{baz}


\definecolor{LLGray}{gray}{0.87}

%%%% NUMBERS %%%%

%% SOLIDITY %%
% Alex's reports
\newcommand{\solASubmitted}{88}
% all of the below are in terms of \solSubmitted
\newcommand{\solAShouldCompile}{16}
\newcommand{\solAValidFixed}{75}
\newcommand{\solAValidDuplicates}{9}
\newcommand{\solAUniqueFixed}{66} % \solAValidFixed - \solAValidDuplicates
\newcommand{\solANotValid}{7} % or unreproducible
\newcommand{\solAUnconfirmed}{4}
\newcommand{\solACurrentConfirmedUnfixed}{2}
\newcommand{\solACurrentInProgress}{1}

%% Rijnard's reports
\newcommand{\solRSubmitted}{2}
\newcommand{\solRUniqueFixed}{2}

\newcommand{\solTotalReported}{90} % \solASubmitted + \solRSubmitted
\newcommand{\solUniqueFixed}{69} % \solAUniqueFixed (66) + \solRUniqueFixed (2) 
\newcommand{\solUniqueConfirmed}{2} % \solACurrentConfirmedUnfixed (2). Still waiting on \solACurrentInProgress (1) , not sure what in-progress is, but presumably confirmed.
\newcommand{\solUniqueFixedOrConfirmed}{71} % \solUniqueFixed (68) + \solUniqueConfirmed (3)

%% MOVE %%
\newcommand{\movUniqueFixed}{12}
\newcommand{\movUniqueConfirmed}{2}
\newcommand{\movUniqueFixedOrConfirmed}{14}

%% FE %%
\newcommand{\feSubmitted}{63}
\newcommand{\feUniqueFixed}{41}
\newcommand{\feUniqueConfirmed}{8}
\newcommand{\feUniqueFixedOrConfirmed}{49}

%% Zig %%
\newcommand{\zigUniqueConfirmed}{2}
\newcommand{\zigUniqueFixedOrConfirmed}{2}

%% ALL %%
\newcommand{\allUniqueFixed}{122} % \solUniqueFixed (68) + \movUniqueFixed (12) + \zigUniqueFixed (0) + \feUniqueFixed (41)
\newcommand{\allUniqueConfirmed}{15} % \solUniqueConfirmed (3) + \movUniqueConfirmed (2) + \zigConfirmed (2) + \feUniqueConfirmed (8)
\newcommand{\allUniqueFixedOrConfirmed}{136} % \solUniqueFixedOrConfirmed (70) + \movUniqueFixedOrConfirmed (14) + \zigUniqueFixedOrConfirmed (2) + \feUniqueFixedOrConfirmed (49)

\newcommand{\cmark}{\ding{51}}%
\newcommand{\xmark}{\ding{55}}%
\newcommand{\acirc}{\textbf{--}}

\newcommand{\ph}[1]{\phantom{#1}}


\newcommand\vtextvisiblespace[1][.5em]{%
  \makebox[#1]{%
    \kern.07em
    \vrule height.3ex
    \hrulefill
    \vrule height.3ex
    \kern.07em
  }% <-- don't forget this one!
}


\tcbset{
%    center title,
    boxrule=0pt,
    left=0pt,
    right=0pt,
    top=0pt,
    bottom=0pt,
%    colback=gray!70,
%    colframe=white,
%    width=.5\columnwidth,
%    boxsep=1pt,
%    arc=0pt,outer arc=0pt,
    }

\definecolor{dkgreen}{rgb}{0,0.5,0}
\definecolor{dkred}{rgb}{0.5,0,0}
\definecolor{gray}{rgb}{0.5,0.5,0.5}
\definecolor{vlgray}{gray}{0.95}
\definecolor{lgray}{gray}{0.7}
\definecolor{bluehighlight}{HTML}{46adb7}
\definecolor{orangehighlight}{HTML}{e3a24d}
\definecolor{redhighlight}{HTML}{ffeef0}
\definecolor{greenhighlight}{HTML}{e6ffed}
\definecolor{grayhighlight}{HTML}{6a737d}
\definecolor{blu}{HTML}{035cc5}

\lstdefinestyle{gostyle}{
  basicstyle=\ttfamily\bfseries\scriptsize\linespread{1.1},
  frameround=tttt,
  frame=single,
  keywordstyle=\color{blue},
  commentstyle=\color{dkred},
  stringstyle=\color{dkgreen},
% https://tex.stackexchange.com/questions/329533/listings-package-lstinline-command-has-strange-spacing-behaviour-after-double-qu
  keepspaces=true,              % very important to preserve space after ' in footnote
  numbers=left,
  breaklines=true,
  otherkeywords={::=},
  numberstyle=\ttfamily\footnotesize\color{gray},
  stepnumber=1,
  numbersep=8pt,
%  backgroundcolor=\color{vlgray},
  backgroundcolor=\color{white},
  tabsize=4,
  showspaces=false,
  showstringspaces=false,
  xleftmargin=.18in,
  captionpos=b,
  escapeinside={(?}{?)},
%  showstringspaces=true
%  escapeinside={$}{$}
}


%%
%% Submission ID.
%% Use this when submitting an article to a sponsored event. You'll
%% receive a unique submission ID from the organizers
%% of the event, and this ID should be used as the parameter to this command.
%%\acmSubmissionID{123-A56-BU3}

%%
%% The majority of ACM publications use numbered citations and
%% references.  The command \citestyle{authoryear} switches to the
%% "author year" style.
%%
%% If you are preparing content for an event
%% sponsored by ACM SIGGRAPH, you must use the "author year" style of
%% citations and references.
%% Uncommenting
%% the next command will enable that style.
%%\citestyle{acmauthoryear}

%%
%% end of the preamble, start of the body of the document source.
\begin{document}
\lstset{style=gostyle}

%%
%% The "title" command has an optional parameter,
%% allowing the author to define a "short title" to be used in page headers.
\title{Making No-Fuss Compiler Fuzzing Effective}

%%
%% The "author" command and its associated commands are used to define
%% the authors and their affiliations.
%% Of note is the shared affiliation of the first two authors, and the
%% "authornote" and "authornotemark" commands
%% used to denote shared contribution to the research.



%%
%% By default, the full list of authors will be used in the page
%% headers. Often, this list is too long, and will overlap
%% other information printed in the page headers. This command allows
%% the author to define a more concise list
%% of authors' names for this purpose.
\renewcommand{\shortauthors}{us folks}

%% Table shortcuts
\newcommand{\mr}[2]{\multirow{#1}{*}{#2}}
\newcommand{\mc}[3]{\multicolumn{#1}{#2}{#3}}

%% comments
\newcommand{\clg}[1]{\textcolor{blue}{#1}}
\newcommand{\rvt}[1]{\textcolor{purple}{#1}}
\newcommand{\kj}[1]{\textcolor{olive}{#1}}

%%
%% The abstract is a short summary of the work to be presented in the
%% article.
\begin{abstract}
Developing a bug-free compiler is difficult; modern optimizing compilers are among the most complex software systems humans build.  Fuzzing is one way to identify subtle compiler bugs that are hard to find with human-constructed tests.  Grammar-based fuzzing, however, requires a grammar for a compiler's input language, and can miss bugs induced by code that does not actually satisfy the grammar the compiler \emph{should} accept.  Grammar-based fuzzing also seldom uses advanced modern fuzzing techniques based on coverage feedback. However, modern mutation-based fuzzers are often ineffective for testing compilers because most inputs they generate do not even come close to getting past the parsing stage of compilation.   This paper introduces a technique for taking a modern mutation-based fuzzer (AFL in our case, but the method is general) and augmenting it with operators taken from \emph{mutation testing}, and program splicing. We conduct a controlled study to show that our hybrid approaches significantly improve fuzzing effectiveness qualitatively (consistently finding unique bugs that baseline approaches do not) and quantitatively (typically finding more unique bugs in the same time span, despite fewer program executions). Our easy-to-apply approach has allowed us to report more than 100 confirmed and fixed bugs in production compilers, and found a bug in the Solidity compiler that earned a security bounty.  
\end{abstract}

\begin{CCSXML}
<ccs2012>
<concept>
<concept_id>10011007.10010940.10010992.10010998.10011001</concept_id>
<concept_desc>Software and its engineering~Dynamic analysis</concept_desc>
<concept_significance>500</concept_significance>
</concept>
<concept>
<concept_id>10011007.10011074.10011099.10011102.10011103</concept_id>
<concept_desc>Software and its engineering~Software testing and debugging</concept_desc>
<concept_significance>500</concept_significance>
</concept>
</ccs2012>
\end{CCSXML}

\ccsdesc[500]{Software and its engineering~Dynamic analysis}
\ccsdesc[500]{Software and its engineering~Software testing and debugging}

\keywords{fuzzing, compiler development, mutation testing}


\maketitle


\section{Introduction}

Compilers are notoriously hard to test, and modern optimizing
compilers tend to contain many subtle bugs.  Compiler bugs can have
serious consequences, including, potentially, the introduction of
security vulnerabilities that cannot be detected without knowledge of a compiler flaw~\cite{CompBug}.   The
literature on compiler testing is extensive~\cite{chen2020survey}.

\small
\begin{table*}
\centering
\begin{tabular}{p{35mm}p{31mm}p{30mm}p{67mm}}
\toprule
\bf Technique & \bf Tool & \bf Requirements    & \bf Weaknesses \\
              &          & \bf from Developers &                \\
% \midrule
\rowcolor{LLGray}
Custom tool (e.g. Csmith)  
& Custom tool 
& None 
& Extremely labor-intensive, potentially years of work
\\
%\hline
Grammar-based              
& Grammar-based fuzzer             
& Usable grammar 
& Needs tuning, many bugs not in scope 
\\
% \hline
\rowcolor{LLGray}
``No-fuss'' mutation-based 
& Off-the-shelf fuzzer \newline (e.g., AFL) 
& Corpus of examples 
& Inefficient, has trouble hitting ``deep'' bugs; may focus on
                        least interesting bugs
  \\
\bottomrule
\end{tabular}
\caption{Compiler Fuzzing Techniques}
\label{tab:techniques}
\end{table*}
\normalsize


As McKeeman's~\cite{Differential} widely cited paper suggests, one core approach to testing compilers is
based on
the generation of \emph{random programs}.  Csmith~\cite{csmith} is perhaps the most prominent
example of this method.  Building a tool such as Csmith
is a heroic effort, requiring considerable expertise and
development time.  Csmith itself is over 30KLOC, much of it complex
and with a lengthy development history.  Csmith is focused on a
single, albeit extremely important, language: C.  Building a tool like
Csmith for a new programming language is not within the scope of most
compiler projects, even major ones.  For instance, to our knowledge
there is \emph{no} useful tool for generating random Rust programs
(none seems to be
prominently featured in {\tt rustc} testing).  Rust
is primarily (or perhaps \emph{only}) fuzzed at the whole language
level
(\url{https://github.com/dwrensha/fuzz-rustc/blob/master/fuzz_target.rs}) by
using a wrapper around libFuzzer, a tool with no knowledge of Rust, to randomly modify \emph{a set of supplied Rust
programs}.  Similarly, the {\tt solc} compiler, used for most smart contracts on
the Ethereum blockchain, is fuzzed using methods similar to those used for
Rust\footnote{Creating a
grammar-based fuzzer has been an open issue for Solidity since August
of 2020 (\url{https://github.com/ethereum/solidity/issues/9673}).}; we
call these approaches, based on mutating a starting set of
programs, \emph{no-fuss} fuzzing.  

\begin{sloppypar}
 Most compiler projects, even large ones, do not have a team
of spare random testing and compiler/language experts available, so the construction of
Csmith-like tools is out of the question.  This means that the only
way to generate valid programs \emph{from scratch} is to use a tool that takes as
input a \emph{grammar}, and generates random outputs
satisfying the grammar.   However, such an approach has multiple problems.
First, in many cases the programs produced by a grammar, without
extensive attention to tuning the probabilities of productions, etc., will be mostly uninteresting.
Csmith is successful in part because of the use of numerous heuristics
to generate interesting code.  Second, the grammar of a language alone
seldom provides guidance in avoiding simple errors that cause programs
to be rejected without exploring interesting compiler behavior; e.g., forcing identifiers to be
defined before they are used.  Third, many interesting bugs can \emph{only}
be exposed by programs that do not satisfy a language's supposed grammar, due to differences between a formal grammar and the actual
parser used in a compiler, or other subtle implementation details.
Salls et al.~\cite{Salls2021TokenLevel} found that many bugs could not be discovered using a
grammar-based generator.
Finally, a usable grammar simply may not be available, especially as the
tools will expect a grammar in a particular format (e.g. antlr4), and may add
restrictions on the structure of the grammar.  In the early stages,
many programming language projects lack a stable, well-defined
grammar in any formal, stand-alone, notation.  An ad-hoc ``grammar'' used by the compiler implementation may be the
only grammar around.  Thus, while grammar-based compiler
testing has sometimes been extremely successful~\cite{LangFuzz}, few compilers are actually
extensively tested that way.
\end{sloppypar}

Unfortunately, ``no-fuss''
fuzzing must make use of off-the-shelf \emph{fuzzing} tools,
originally designed to find security vulnerabilities in inputs treated
largely as byte-streams.  No-fuss
fuzzing therefore
suffers from two major drawbacks:

\begin{enumerate}
\item The methods used by fuzzers to mutate inputs tend to take code that exercises interesting
  compiler behavior, and transform it into code that is rejected by
  the parser.  This is inefficient, and makes it
  almost impossible to find bugs requiring a sequence of subtle
  modifications.
  \item Bugs are often
    found via very un-human-like inputs.
  \end{enumerate}

  Combined together, these problems tend to make most compiler fuzzing
  performed in practice inefficient in
  terms of finding bugs and prone to find less interesting bugs.
  Table~\ref{tab:techniques} summarizes the existing widely-used
  compiler fuzzing techniques and their weaknesses.

  Given that ``no-fuss'' fuzzing is widely used in large projects and
  may be the \emph{only} option available in
  practice to small compiler projects, improving the effectiveness of no-fuss fuzzing is an obvious way to practically improve compiler
  testing.  Ideally, such improvements would not require \emph{any}
  additional effort on the part of developers.

  This paper proposes one such improvement, based on changing the way
  in which general-purpose fuzzers modify (mutate) inputs.  We augment
  the set of primarily byte-based changes made by such tools with a large number of
  modifications drawn from the domain of \emph{mutation testing},
  which only modifies code in ways likely to preserve desirable
  properties---like the ability to get through a parser. Figure~\ref{lst:sol-exemplar}
  is one such input generated by our approach, yielding a syntactically well-formed program that
  triggers deeper behavior in the compiler's optimization routines.

\begin{figure}[h!]
\begin{lstlisting}[basicstyle=\scriptsize\ttfamily,numbers=none,xleftmargin=0.7em,xrightmargin=.7em]
(?{\color{blu}contract}?) (?{\color{dkgreen}C}?) {
  (?{\color{blu}function}?) (?{\color{dkgreen}fun\_x}?) () (?{\color{blu}public}?) {}
  (?{\color{blu}function}?) (?{\color{dkgreen}fun\_y}?) () (?{\color{blu}public}?) {}
  (?{\color{blu}function}?) (?{\color{dkgreen}f}?)() (?{\color{blu}public}?)
  {
    int h(?{\color{dkred}=}?)true(?{\color{dkred}\verb|?|}?)1(?\color{dkred}\verb|:|?)3(?{\color{dkred};}?)
  }
  (?{\color{blu}function}?) () r(?{\color{dkred}=}?)true(?{\color{dkred}\verb|?|}?)fun_x(?{\color{dkred}\verb|:|}?)fun_y(?{\color{dkred};}?)
}
\end{lstlisting}
\caption{An example of an early crash-inducing Solidity program found
  with our approach (the bug was submitted and fixed). The combination of expressions and function declarations trigger complex behavior in an optimization routine that attempts to deduplicate low level code blocks.}
\label{lst:sol-exemplar}
\end{figure}

  We evaluate
  our technique on four real-world compilers, and show that it
  significantly improves the mean number of distinct compiler bugs
  detected.  We have reported more than
  100 previously undiscovered 
  bugs, subsequently fixed, and received a bug bounty for
  our efforts.  In the longest-running campaign, that targeting the
  {\tt solc} compiler for Solidity code, we were the first to
  report a large number of serious bugs, despite extensive
  fuzzing  performed by the developers, OSS-Fuzz,
  and external contributors.  Our tool, based on Google's release of AFL, is available as open source at
  \url{https://github.com/agroce/afl-compiler-fuzzer}, and to date has
  more than 70 stars and multiple forks.

\section{Mutation-Testing-Based Compiler Fuzzing}



\subsection{Mutation-Based Fuzzing}

One use of the term ``mutation'' appears in the context of \emph{mutation-based} fuzzing~\cite{ArtFuzz}, the primary random testing approach used by many compiler projects, as discussed above.  Again, we note that there are two basic kinds of compiler testing based on the generation of random inputs to a compiler.  One, in recent years paradigmatically expressed in the Csmith tool~\cite{csmith}, works by using a grammar and/or deep knowledge of the language accepted by the compiler, to generate progams to test the compiler.  This is sometimes called \emph{generative} fuzzing.  Generative fuzzing can be very effective, but often requires expert tuning of a large, sophisticated tool, and at minimum requires having a suitable usable grammar for the language of the compiler.  In practice, many compiler projects do not employ generative fuzzing for practical reasons.

A second approach, and the only approach widely used in many major compiler projects, is to use an off-the-shelf fuzzer, such as is used to find security vulnerabilities (e.g., the ubiquitous American Fuzzy Lop (afl) \url{https://github.com/google/AFL}) or libFuzzer, and a \emph{corpus} of example programs, such as the set of regression tests for the compiler or a set of real-world programs.  A fuzzer such as AFL operates by executing the program under test (here, the compiler) on inputs (initially those in the corpus), using instrumentation to determine code coverage in the compiler for each executed input.  The fuzzer then takes inputs that look interesting and adds them to a \emph{queue}.  The basic loop is then to take some input from the queue, \emph{mutate} it by making some essentially random change (e.g., flipping a single bit, or removing a random chunk of bytes), execute the new, mutated input under instrumentation, and add the new input to the queue if it seems ``interesting'' --- typically, if it hits some kind of coverage target that has not previously been hit.  The details of selecting inputs from the queue and determining how to mutate an input vary widely, and improving the effectiveness of this basic approach has been a major topic of recent software testing and security research.  However, the basic strategy usually still fits into a simple basic model:

\begin{enumerate}
\item Select an input from the queue.
\item Mutate that input in order to obtain a new input.
\item Execute the new input, and if it is deemed interesting, add it to the queue.
\item Go back to the first step.
\end{enumerate}

Any inputs that crash the compiler in step 3 are reported to the user.  Using such a fuzzer is often extremely easy, involving no more work than 1) building the compiler with special instrumentation and 2) finding a set of initial programs to use as a corpus.  Even compiling with instrumentation can be optional; some fuzzers (including AFL) can use QEMU to fuzz arbitrary binaries.  However, for compilers, it is usually best if possible to rebuild the compiler, since QEMU-based execution is much slower, and compilers are slow enough to seriously degrade fuzzing throughput.


Our work focuses on improving step 2 of this process, in a way that is agnostic to how the details of the other aspects of fuzzing are implemented.  In particular, the problem with most approaches to mutation in the literature, for compiler fuzzing, is that changes such as byte-level-transformations almost always take compiling programs that exercise interesting compiler behavior, and transform them into programs that don't make it past early stages of parsing.  Alternative approaches to what are called ``havoc''-style mutations tend to involve solving constraints or following taint, which in the case of compilers tends to be ineffective, since the relationships to be preserved are quite complex, and implemented in complex code.  A second common approach, providing a \emph{dictionary} of meaningful byte sequences in a language, is both burdensome on compiler developers (though less so than providing a full grammar), and limited in effectiveness: a dictionary cannot, for example, help the fuzzer delete meaningful sub-units of code, such as statements or blocks.

We propose a novel way to produce a much larger number of useful, interesting mutations for source code, without paying an analytical price that makes fuzzing practically infeasible for compilers, and without requiring \emph{any} additional effort on the part of compiler developers.

\subsection{Mutation Testing}

A different use of the term ``mutation'' appears in the field of mutation testing.  Mutation testing~\cite{MutationSurvey,budd1979mutation,demillo1978hints} is an approach to evaluating and improving software tests that works by introducing small syntactic changes into a program, under the assumption that if the original program was correct, then a program with slightly different semantics will be incorrect, and should be flagged as such by effective tests.  Mutation testing is now widely used in software testing research, and is used to varying degrees in industry at-scale and for especially critical software development~\cite{mutKernel,mutGoogle,mutFacebook}.

A mutation testing approach is defined by a set of mutation operations.  Such operations vary widely in the literature, though a few, such as deleting a small portion of code (such as a statement) or replacing arithmetic and relational operations (e.g., changing {\tt +} to {\tt -} or {\tt ==} to {\tt <=}), are very widely used.  Most mutation testing tools parse the code to be mutated, and many do not work on code that does not parse.  However, recently there has been a proposal to perform mutation testing using truly purely syntactic operations, defined by a set of regular expressions implemention operations~\cite{regexpMut}.  Rather than taking a program, per se, this approach simply takes ``code-like'' text and produces a set of variants that, if the original text is compiling source code, will include most common mutations.  The essence of this approach to mutation testing, which can be applied to ``any language,'' is essentially a transformation from arbitrary bytes to arbitrary bytes that, if the original bytes are ``code-like'' will tend to preserve the property of being ``code-like.''


\subsection{Combining Both Forms of Mutation}

Our approach is, in essence, simple.  We add a set of mutations to the repertoire of a mutation-based fuzzer, for use in compiler fuzzing.  These mutations are either traditional mutation operators from the mutation testing domain or inspired by traditional mutation operators, but with changes made to satisfy the needs of fuzzing.  The key point is that, unlike most changes made by mutation-based fuzzers, these mutations are likely to take interesting code inputs and preserve the property, e.g., that the input will get through a parser or trigger interesting optimizations.  The tendency to preserve such properties is natural, since the basis of mutation testing is to take an existing program and produce a set of new, similar programs, by applying mutation operators.  If most mutation operators tended to produce uninteresting code that doesn't even compile, mutation testing would not be of use to anyone.  Moreover, because our approach is based on the idea of a ``universal'' mutation tool~\cite{regexpMut}, the mutation operators used are generally language-agnostic, and useful for fuzzing any programming language that syntactically resembles common languages (under which we include not only C-like languages, but even LISP-like languages based on s-expressions).

\subsection{Limitations}

The most important limitation for the mutation-testing-based approach is that if compiler \emph{crashes} are mostly uninteresting, fuzzing of this kind will probably not be very useful.  This applies, of course, to all AFL-style fuzzing, not just to fuzzing using the technique proposed in this paper.  For example, C and C++ include a large variety of undefined behaviors.  Code that crashes a C or C++ compiler, but that includes (unusual) undefined behavior may well be ignored by developers.  Csmith~\cite{csmith} devotes a great deal of effort to avoiding generating code that falls outside the ``ineresting'' part of the language.  On the other hand, many languages more recent than C and C++ attempt to provide a more ``total'' language where, while a program may be considered absurd by a human, fewer (or no) programs are undefined in the sense that C and C++ use the term.  For example, smart contract languages such as those studied in this paper, generally aim to make all programs that compile well-defined, or at least minimize the problem to more managable cases such as order of evaluation of sub-expressions.  Similarly, Rust code without use of {\tt unsafe} should not crash the compiler, and any such crashes indicate possible bugs in the Rust compiler or type system.  For most more recent languages, and some older languages such as Java, a program that crashes the compiler is, in general, likely of interest to compiler developers.  However, the proposed technique will be much more limited in effectiveness for C and C++ compilers.
\section{Implementation and Example Operations}

\subsection{Fast or Smart?}

\subsubsection{Fast String-level Approximation of Mutation Operators}

\subsubsection{Parser Parser Combinator-Based Intelligent Mutation}
\section{Evaluation}
\label{eval}

We ran a series of controlled experiments to evaluate the effectiveness of two
strategies based on our approaches in Sections~\ref{strat-fast-string-level}
and~\ref{strat-syntax-aware} for improving on ``no-fuss'' compiler fuzzing. Our
main goal is to answer to what extent these low-effort strategies demonstrate
significant benefit in the domain of compiler fuzzing, and how they influence
fuzzer behavior and performance.

Section~\ref{exp-setup} describes our experimental setup.
Section~\ref{exp-results} summarizes our results, which compares our strategies
against stock AFL and AFL++ fuzzers on four actively-developed compilers. The
compiler languages we cover include
smart contract languages
Solidity,\footnote{https://docs.soliditylang.org/}
Move,\footnote{https://move-book.com/} and Fe\footnote{https://fe-lang.org/},
and the trending Zig
language.\footnote{https://ziglang.org/}

\subsection{Experimental Setup}
\label{exp-setup}



\noindent 
\textbf{Input corpora.} For solidity, all \texttt{.sol} files in
\texttt{test/libsolidity}. For Move, all \texttt{.diem} files in the
repository. For \texttt{Fe}, {\color{red} all \texttt{.fe} files in
repository}. AFL starts of preprocessing inputs based on coverage and ignores uninteresting ones. {\color{red} say how many for each proj}

For the template splicing technique, we start off with a noop input and
gradually generate \texttt{(template, fragment)} pairs, synthesized on-demand.
After generating pairs,wWe removed all large inputs in the \texttt{(template, fragment)} corpus (> 4KB). {\color{red} say how many of these}.

% Data lives here:
% Solidity: /home/rijnard/0-experiments-feb/sol-programs                         @ 686b62b585d686f08fe2f8d586b8474c133dce2f + cherry-pick
%           /home/rijnard/0-experiments-feb/comby-mutation-server/fragments
% Move:     /home/rijnard/0-experiments-feb/move-programs                        @ bfb6b09715894b3c436919bf2e718b6ae0fcba9f (double check)
%           /home/rijnard/0-experiments-feb/comby-mutation-server-move/fragments
% FE        /home/rijnard/0-experiments-feb/fe-programs                          @ 1ea2206e3d10e77163f1a01bee05088358d8ef23
%           /home/rijnard/0-experiments-feb/comby-mutation-server-fe/fragments


\subsection{Results}
\label{exp-results}

\begin{table*}
\centering
\begin{tabular}{llrrrrrrcr}
\toprule
                    \bf Project      & \bf Configuration                           & \mc{3}{c}{\bf Unique Bugs}        & \bf Avg Execs  & \bf Avg Paths    & \bf Avg Bitmap    & \mc{2}{c}{\bf Queue}            \\
                                     &                                             & Avg     & Min       & Max         & (Millions)     & (K)              & Cvg (\%)          & Compiles (K)     & Unique Errs \\
\midrule
                    \mr{4}{Solidity} & \tt \small      AFL-baseline                &  3.69   & 1         &  6          & 35.8           & 12.0             & 54.34\ph{a}       & 2.89             & ????          \\ 
                                     & \tt \small      AFL++ 3.15a                 &  5.63   & 1         & 10          & 56.9           &  8.8             & 20.58$^\dagger$   & 3.80             & ????          \\ 
                                     & \tt \small      text-mutation               &  7.81   & 7         & 11          & 30.3           & 14.3             & 55.65\ph{a}       & 5.48             & ????          \\ 
                                     & \tt \small      splice-mutation             & 11.81   & 7         & 14          & 16.0           & 16.8             & 57.33\ph{a}       & 5.24             & ????          \\ 
\midrule
                    \mr{4}{Move}     & \tt \small      AFL-basline                 & 7.19    & 6         & 8           & 56.9           & 4.9              & 63.23\ph{a}       &                  &               \\ 
                                     & \tt \small      AFL++ 2.54b                 & ????    & ?         & ?           & ????           & ???              & ?????             &                  &               \\ 
                                     & \tt \small      text-mutation               & 8.31    & 7         & 9           & 61.2           & 6.0              & 62.27\ph{a}       &                  &               \\ 
                                     & \tt \small      splice-mutation             & 6.06    & 5         & 7           &  7.2           & 5.0              & 63.18\ph{a}       &                  &               \\ 
\midrule
                    \mr{4}{Fe}       & \tt \small      AFL-baseline                & 6.56    & 5         & 7           & 24.5           & 3.5              & 27.91\ph{a}       &                  &               \\ 
                                     & \tt \small      AFL++ 2.64c                 & 6.44    & 5         & 8           & 22.6           & 3.4              & 27.76\ph{a}       &                  &               \\ 
                                     & \tt \small      text-mutation               & 6.50    & 5         & 7           & 17.9           & 3.3              & 27.84\ph{a}       &                  &               \\ 
                                     & \tt \small      splice-mutation             & 6.94    & 6         & 9           &  5.0           & 2.6              & 27.83\ph{a}       &                  &               \\ 
\midrule
                    \mr{3}{Zig}      & \tt \small      AFL-baseline                & ????    & ?         & ??          & 2.2            & 3.3              & 40.99\ph{a}       &                  &               \\ 
                                     & \tt \small      text-mutation               & ????    & ?         & ??          & 2.1            & 3.3              & 40.95\ph{a}       &                  &               \\ 
                                     & \tt \small      splice-mutation             & ????    & ?         & ??          & 1.3            & 3.9              & 41.82\ph{a}       &                  &               \\ 
\bottomrule
\end{tabular} 
        \caption{Main results. We fuzzed each project for 16 trials (24 hours per trial) in different configurations: \texttt{baseline-AFL}, \texttt{AF++},  \texttt{text-mutation}, and \texttt{splice-mutation}.
\texttt{baseline-AFL} is stock \texttt{AFL}; \texttt{AFL++} is a community-driven effort that enhances stock AFL. \texttt{text-mutation} applies mutation operators (textual find-replace patterns) with a probability of 75\% on every fuzzed input. Sock AFL manipulates the input witht the remainder, 25\% of the time. \texttt{splice-mutation} is a hybrid approach that (1) applies mutation operators as in \texttt{text-mutation} with probability 33\%; (2) synthesizes a syntax-aware input with template (splice) 33\% of the time, and (3) uses stock AFL for the remainding 34\%. {\color{red} TODO: summarize results once flush.}}
\label{tab:results}
\end{table*}




\section{Non-experimental Fuzzing Campaigns}
\label{real-world}

\begin{table}
\caption{Fuzzing campaign results for real world bugs. {
\cmark~is \textbf{fixed} bugs. \clock~is \textbf{confirmed but unfixed bugs}. 
\acirc~is \textbf{duplicate bug reports}.
\textbf{Total} is the number of true, unique bugs reported and acknowledged.}
}
\centering
\begin{tabular}{lrr|rrr}
\toprule
                    \bf Project       & \bf Length & \bf Total                        & \cmark            & \clock                  & \acirc                 \\
\midrule
                    Solidity          & 20mo 30d      & \solUniqueFixedOrConfirmed      & \solUniqueFixed   & \solUniqueConfirmed     & \solAValidDuplicates   \\
                                      % Fuzz: about a week Reports: Jan 27 to Feb 9 2021.
                    Move              & 20d        & \movUniqueFixedOrConfirmed       & \movUniqueFixed   & \movUniqueConfirmed     & 0                      \\
                                                                                                                                    % 63 submittd, 49 are fixed or confirmed, are the other 14 wrong or dups?
                    Fe                & 9mo 6d      & \feUniqueFixedOrConfirmed        & \feUniqueFixed    & \feUniqueConfirmed      & \feValidDuplicates \\
                                      % Fuzz: about a week. Reports: 1 day
                    Zig               & 7d         & \zigUniqueFixedOrConfirmed       & \zigUniqueFixed   & \zigUniqueConfirmed     & 0                      \\
\midrule
                    All               &            & \allUniqueFixedOrConfirmed       & \allUniqueFixed   & \allUniqueConfirmed     & \allValidDuplicates    \\
\bottomrule
\end{tabular}

\label{tab:campaign-fixes}
\end{table}

In addition to our controlled experiments fuzzing a version of each compiler before we reported any bugs, we ran real fuzzing campaigns, updating the compiler versions as new commits were made, and adjusting our corpus to include new tests, of various durations, on each compiler.  Two of these campaigns are ongoing (for Solidity and Fe) and have been well-supported and well-received by the compiler teams.

\begin{sloppypar}
  Perhaps the most important evidence of the real world effectiveness is that we fuzzed the Solidity compiler for over a year with our approach, and in that time reported a large number of otherwise unreported bugs that have been fixed; we submitted our most recent bug (as of this writing) on 11/2/21.  Prior to and during our campaign, Solidity had been fuzzed heavily using AFL, using a dictionary, by the developers and by external contributors, and has been on OSS-Fuzz since the first quarter of 2019 (\url{https://blog.soliditylang.org/2021/02/10/an-introduction-to-soliditys-fuzz-testing-approach/}).  While no grammar-based approach has been applied to Solidity as a whole, the Yul IL has been fuzzed using Google's libprotobuf-mutator library (\url{https://github.com/google/libprotobuf-mutator}).  Despite competing with these efforts, and never devoting more resources to the fuzzing than 3-4 docker container hosted instances of our fuzzing tool, running on a high-end laptop, we believe that our campaign was the largest single source of fuzzing-discovered bugs in the compiler during our campaign.  The campaign was awarded a security bounty of \$1,000 USD in Ethereum for discovery of a bug with potential security implications (\url{https://github.com/ethereum/solidity/issues/8368}) (and, it was noted, for the general effectiveness of the fuzzing), and the Solidity team encouraged and aided our efforts, once it was clear that the approach was very useful in exposing subtle bugs not otherwise discovered.  Because Solidity bug triage is very well supported, we can add an additional measure of the effectiveness of our approach in finding bugs that involve ``realistic'' code.  After October of 2020, the Solidity team began adding a ``should compile without error'' label to submitted bugs that involved legal code the compiler rejects.  Of the 38 bugs we submitted since that date, 15 (nearly 40\%) have involved correct but rejected (via a crash) code.  Such bugs are inherently harder to find and usually more interesting than those where a compiler crashes rather than reporting an error when given invalid code.
\end{sloppypar}

A second long-term fuzzing effort was directed at the Fe language, a Rust/Python-like alternative to Solidity for writing Ethereum contracts.  Fe is an experimental language, and the project has far fewer resources than Solidity to devote to testing.  Fe developers received this effort warmly, and quickly made some changes to the Fe compiler to make AFL fuzzing more effective (by crashing when Fe caused the Yul backend to fail).  Using our approach, we were able to provide the project with high-quality fuzzing very early in the lifetime of an experimental compiler project.  We speculate that better ``no-fuss'' fuzzing could expose language corner cases early in the implementation of a compiler, avoiding having to make costly changes later, when more code depends on erroneous implementation assumptions or (even more disastrously) poor language design choices.  Some of our bug reports triggered lengthy discussions in the issue of a language or compiler design foundational decision.  Due to the small size of the Fe team, 8 of our reported bugs have not been analyzed yet and confirmed as valid.  The most recent of these was submitted (as of this writing) on 9/25/21.  The Fe effort was sufficiently influential that it was invoked in discussions of the long-term strategies for building language-customized fuzzing for Fe (e.g., \url{https://github.com/ethereum/fe/pull/578#pullrequestreview-790913799}).

At the time we fuzzed Facebook's \texttt{Move} compiler, the project had been fuzzing various components, but less so the compiler itself.
The majority of bugs reported were quickly confirmed and fixed, and developers expressed interest in incorporating our approach into CI (\url{https://github.com/diem/diem/issues/7384#issuecomment-769443728}).


We ran a shorter, less-intensive campaign on the \texttt{Zig} compiler.  The
\texttt{Zig} compiler continues to be under heavy development, and a small team
of maintainers are prioritizing efforts to rewrite the components where we found
bugs.

\begin{sloppypar}
  We additionally reported 5 (mostly parser crash) bugs in the SPIN model checker, which were all fixed (e.g., \url{https://github.com/nimble-code/Spin/commit/7f364a1b174f08e9ede49e342f411e209af26a84}).
  \end{sloppypar}

\section{Related Work}

Research on compiler testing, as noted in the introduction, has been an important subfield overlapping compiler development and design and software engineering and testing, for many years.  Chen et al. summarize much of this work in a recent survey~\cite{chen2020survey}.

To our knowledge, very little work has appeared targeting the problem this paper addresses: improving the ability of general-purpose fuzzers to find (interesting) bugs in compilers.  The recent work of Salls et al.~\cite{Salls2021TokenLevel}, however, specifically aims to improve general-purpose fuzzer performance on compilers and interpreters.  Their approach, which they call ``token-level fuzzing'' essentially produces a hybrid level in between grammar-based generation and ``byte-level'' mutation-based fuzzing.  The core of their idea is to replace the largely byte-level mutations of AFL etc. with mutations at the \emph{token} level of a grammar.  They summarize the idea as ``valid tokens should be replaced with valid tokens''~\cite{Salls2021TokenLevel}.  In a sense, this extends the idea of using a dictionary, but with important changes:  token-level fuzzing \emph{only} applies token-level, not byte-level mutations, but also adds the composition of multiple token additions and substitutions to the set of single-step mutations.  Token-level fuzzing is an attractive and useful idea, somewhat orthogonal to our approach.  However, unlike our approach, token-level fuzzing does not apply AFL's havoc operations, so some bugs are simply not possible to find using token-level fuzzing (e.g., ones involving injecting unprintable characters in strings, including our Solidity bug earning a security bounty).  In this sense, token-level fuzzing has some of the limitations of grammar-based generation.  Token-level fuzzing also provides little help to a fuzzer in deleting large chunks of code, since this often would require a very large number of token operations, though the approach does include a way to copy statements from one input to another.  Finally, token-level fuzzing requires using a lexer to find all tokens in input seeds, and if tokens not in those seeds would be useful, developers must provide any additional tokens.  This requires modifying the fuzzing workflow to add token pre-processing, and is no longer strictly \emph{no-fuss}, though in practice the change is fairly small (which is also true of our approach when comby pre-processess the corpus).
\section{Conclusions}

Using automated methods to find bugs is an important part of modern
compiler development.  For a variety of reasons, the use of
off-the-shelf fuzzers, especially AFL, is the most widely used such
approach.  Mutation-based fuzzers, however, were originally used to
test binary formats, and their heritage limits their effectiveness for
fuzzing compilers.  Most solutions to this problem either place significant
burden on compiler developers, or are ineffective, or both.  We show
that using ideas from mutation testing it is possible to significantly
improve AFL-based fuzzing of compilers without forcing developers to
provide grammars or a dictionary.  One of our two configurations
performed best for all compilers we investigated, in experiments,
sometimes dramatically so (finding more than twice as many bugs on
average as the best version of un-augmented AFL).  Using our approach
we reported more than 100 previously unknown bugs, that have been
fixed as a result, in important
compiler projects.

\begin{sloppypar}
{\bf Acknowledgements:}  A portion of this work was
  supported by the National Science Foundation under CCF-2129446.  The
authors would also like to thank our anonymous reviewers and, finally, the
developers of the various compilers we tested, who confirmed and fixed
the bugs we found, and in some cases actively changed the compiler or
its infrastructure to support our efforts.
\end{sloppypar}

\balance

\bibliographystyle{ACM-Reference-Format}
\bibliography{bibliography}

\balance

\end{document}
