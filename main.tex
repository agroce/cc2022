%%
%% This is file `sample-sigconf.tex',
%% generated with the docstrip utility.
%%
%% The original source files were:
%%
%% samples.dtx  (with options: `sigconf')
%%
%% IMPORTANT NOTICE:
%%
%% For the copyright see the source file.
%%
%% Any modified versions of this file must be renamed
%% with new filenames distinct from sample-sigconf.tex.
%%
%% For distribution of the original source see the terms
%% for copying and modification in the file samples.dtx.
%%
%% This generated file may be distributed as long as the
%% original source files, as listed above, are part of the
%% same distribution. (The sources need not necessarily be
%% in the same archive or directory.)
%%
%% The first command in your LaTeX source must be the \documentclass command.
\documentclass[sigplan,review]{acmart}

\usepackage{code}
\usepackage{graphicx}
\usepackage{balance}
\usepackage{multirow}
\usepackage{multicol}

%%% The following is specific to Onward! '21 and the paper
%%% 'Let a Thousand Flowers Bloom: On the Uses of Diversity in Software Testing'
%%% by Alex Groce.
%%%
\setcopyright{acmcopyright}
\acmPrice{15.00}
\acmDOI{10.1145/3486607.3486772}
\acmYear{2022}
\copyrightyear{2022}
\acmSubmissionID{onward21essays-id2-p}
\acmISBN{978-1-4503-9110-8/21/10}
\acmConference[foo]{bar}
\acmBooktitle{baz}


%%
%% Submission ID.
%% Use this when submitting an article to a sponsored event. You'll
%% receive a unique submission ID from the organizers
%% of the event, and this ID should be used as the parameter to this command.
%%\acmSubmissionID{123-A56-BU3}

%%
%% The majority of ACM publications use numbered citations and
%% references.  The command \citestyle{authoryear} switches to the
%% "author year" style.
%%
%% If you are preparing content for an event
%% sponsored by ACM SIGGRAPH, you must use the "author year" style of
%% citations and references.
%% Uncommenting
%% the next command will enable that style.
%%\citestyle{acmauthoryear}

%%
%% end of the preamble, start of the body of the document source.
\begin{document}

%%
%% The "title" command has an optional parameter,
%% allowing the author to define a "short title" to be used in page headers.
\title{No-Fuss Compiler Fuzzing}

%%
%% The "author" command and its associated commands are used to define
%% the authors and their affiliations.
%% Of note is the shared affiliation of the first two authors, and the
%% "authornote" and "authornotemark" commands
%% used to denote shared contribution to the research.



%%
%% By default, the full list of authors will be used in the page
%% headers. Often, this list is too long, and will overlap
%% other information printed in the page headers. This command allows
%% the author to define a more concise list
%% of authors' names for this purpose.
\renewcommand{\shortauthors}{us folks}

%% Table shortcuts
\newcommand{\mr}[2]{\multirow{#1}{*}{#2}}
\newcommand{\mc}[3]{\multicolumn{#1}{#2}{#3}}

%% comments
\newcommand{\clg}[1]{\textcolor{blue}{#1}}
\newcommand{\rvt}[1]{\textcolor{purple}{#1}}
\newcommand{\kj}[1]{\textcolor{olive}{#1}}

%%
%% The abstract is a short summary of the work to be presented in the
%% article.
\begin{abstract}
Developing a bug-free compiler is difficult; modern optimizing compilers are among the most complex software systems humans build.  Fuzzing is one way to identify subtle compiler bugs that are hard to find with human-constructed tests.  Grammar-based fuzzing, however, requires a grammar for a compiler's input language, and can miss bugs induced by code that does not actually satisfy the grammar the compiler \emph{should} accept.  Grammar-based fuzzing also seldom uses advanced modern fuzzing techniques based on coverage feedback. However, modern mutation-based fuzzers are often ineffective for testing compilers because most inputs they generate do not even come close to getting past the parsing stage of compilation.   This paper introduces a technique for taking a modern mutation-based fuzzer (AFL in our case, but the method is general) and augmenting it with operators taken from \emph{mutation testing}, and program splicing. We conduct a controlled study to show that our hybrid approaches significantly improve fuzzing effectiveness qualitatively (consistently finding unique bugs that baseline approaches do not) and quantitatively (typically finding more unique bugs in the same time span, despite fewer program executions). Our easy-to-apply approach has allowed us to report more than 100 confirmed and fixed bugs in production compilers, and found a bug in the Solidity compiler that earned a security bounty.  
\end{abstract}

\begin{CCSXML}
<ccs2012>
<concept>
<concept_id>10011007.10010940.10010992.10010998.10011001</concept_id>
<concept_desc>Software and its engineering~Dynamic analysis</concept_desc>
<concept_significance>500</concept_significance>
</concept>
<concept>
<concept_id>10011007.10011074.10011099.10011102.10011103</concept_id>
<concept_desc>Software and its engineering~Software testing and debugging</concept_desc>
<concept_significance>500</concept_significance>
</concept>
</ccs2012>
\end{CCSXML}

\ccsdesc[500]{Software and its engineering~Dynamic analysis}
\ccsdesc[500]{Software and its engineering~Software testing and debugging}

\keywords{fuzzing, compiler development, mutation testing}


\maketitle


\section{Introduction}

Compilers are notoriously hard to test, and modern optimizing
compilers tend to contain many subtle bugs.  Compiler bugs can have
serious consequences, including, potentially, the introduction of
security vulnerabilities that cannot be detected without knowledge of a compiler flaw~\cite{CompBug}.   The
literature on compiler testing is extensive~\cite{chen2020survey}.

\small
\begin{table*}
\centering
\begin{tabular}{p{35mm}p{31mm}p{30mm}p{67mm}}
\toprule
\bf Technique & \bf Tool & \bf Requirements    & \bf Weaknesses \\
              &          & \bf from Developers &                \\
% \midrule
\rowcolor{LLGray}
Custom tool (e.g. Csmith)  
& Custom tool 
& None 
& Extremely labor-intensive, potentially years of work
\\
%\hline
Grammar-based              
& Grammar-based fuzzer             
& Usable grammar 
& Needs tuning, many bugs not in scope 
\\
% \hline
\rowcolor{LLGray}
``No-fuss'' mutation-based 
& Off-the-shelf fuzzer \newline (e.g., AFL) 
& Corpus of examples 
& Inefficient, has trouble hitting ``deep'' bugs; may focus on
                        least interesting bugs
  \\
\bottomrule
\end{tabular}
\caption{Compiler Fuzzing Techniques}
\label{tab:techniques}
\end{table*}
\normalsize

As McKeeman's~\cite{Differential} widely cited paper suggests, one core approach to testing compilers is
based on
the generation of \emph{random programs}.  The Csmith~\cite{csmith} project is perhaps the most prominent
example of this method.  Builiding a tool such as Csmith
is a heroic effort, requiring considerable expertise and
development time.  Csmith itself is over 30KLOC, much of it complex
and with a lengthy development history.  Csmith is focused on a
single, albeit extremely important, language: C.  Building a tool like
Csmith for a new programming language is not within the scope of most
compiler projects, even major ones.  For instance, to our knowledge
there is \emph{no} useful tool for generating random Rust programs
(certainly none seems to be
prominently featured in {\tt rustc} testing).  As far as we can tell, Rust
is primarily (or perhaps \emph{only}) fuzzed at the whole language
level
(\url{https://github.com/dwrensha/fuzz-rustc/blob/master/fuzz_target.rs}) by
using a wrapper around libFuzzer, a tool with no special knowledge of Rust
syntax or semantics, to randomly modify \emph{a set of supplied Rust
programs}.  Similarly, the {\tt solc} compiler, used for most smart contracts on
the Ethereum blockchain, is not fuzzed using a Csmith-like
generator, but mostly using methods similar to those used for
Rust.  Creating a
grammar-based fuzzer has been an open issue for Solidity since August
of 2020 (\url{https://github.com/ethereum/solidity/issues/9673}).

\begin{sloppypar}
Generating compiler tests by randomly mutating existing programs is widely used by
real-world compiler projects in part because \emph{it is often very easy to
  apply}: we call it ``no-fuss'' compiler fuzzing.  Most compiler projects, even large ones, do not have a team
of spare random testing and compiler/language experts available, so the construction of
Csmith-like tools is out of the question.  This means that the only
way to generate valid programs \emph{from scratch} is to use a tool that takes as
input the \emph{grammar} of a language and generates random outputs
satisfying the grammar.   However, such an approach has multiple problems.
First, in many cases the programs produced by a grammar, without
extensive attention to tuning the probabilities of productions, etc., will be mostly uninteresting.
Csmith is successful in part because of the use of numerous heuristics
to generate interesting code.  Second, the grammar of a language alone
seldom provides guidance in avoiding simple errors that cause programs
to be rejected without exploring interesting compiler behavior; e.g., forcing identifiers to be
defined before they are used.  Third, many interesting bugs can \emph{only}
be exposed by programs that do not satisfy a language's supposed grammar, due to differences between a formal grammar and the actual
parser used in a compiler, or other subtle implementation details.
Salls et al.~\cite{Salls2021TokenLevel} found that many bugs could not be discovered using a
grammar-based generator.
Finally, a usable grammar simply may not be available, especially as the
tools will expect a grammar in a particular format (e.g. antlr4), and may add
restrictions on the structure of the grammar.  In the early stages,
many programming language projects lack a stable, well-defined
grammar in any formal, standalone, notation.  An ad-hoc ``grammar'' used by the compiler implementation may be the
only grammar around.  Thus, while grammar-based compiler
testing has sometimes been extremely successful~\cite{LangFuzz}, few compilers are actually
extensively tested using grammar-based tools.
\end{sloppypar}

Unfortunately, ``no-fuss''
fuzzing must make use of off-the-shelf \emph{fuzzing} tools,
originally designed to find security vulnerabilities in programs, thatessentially treat inputs as an
undifferentiated byte-sequence, with little or no structure.  No-fuzz
fuzzing has found many subtle compiler bugs, but
suffers from two major drawbacks:

\begin{enumerate}
\item The methods used by fuzzers to mutate inputs tend to take code that exercises interesting
  compiler behavior, and transform it into code that is rejected by
  the parser.  This is inefficient, and makes it
  almost impossible to find bugs requiring a sequence of subtle
  modifications.
  \item Second, when such fuzzers do find bugs, the bugs are often
    found in particularly un-humanlike inputs.
  \end{enumerate}

  Combined together, these problems make most compiler fuzzing
  performed in practice, even on major projects, both inefficient in
  terms of finding bugs and perhaps prone to find bugs that are not
  the most important and interesting compiler bugs.
  Table~\ref{tab:techniques} summarizes the existing widely-used
  compiler fuzzing techniques and their weaknesses.

  Given that ``no-fuss'' fuzzing is widely used in large projects and
  may be the \emph{only} option available in
  practice to small compiler projects, improving the effectiveness of no-fuss fuzzing is an obvious way to practically improve compiler
  testing.  Ideally, such improvements would not require \emph{any}
  additional effort or change the workflow of existing compiler
  fuzzing setups.

  This paper proposes one such improvement, based on changing the way
  in which general-purpose fuzzers modify (mutate) inputs.  We augment
  the set of primarily byte-based changes made by such tools with a large number of
  modifications drawn from the domain of \emph{mutation testing},
  which only modifies code in ways likely to preserve desirable
  properties --- such as the ability to get through a parser.  We evaluate
  our technique on four real-world compilers, and show that it
  significantly improves the mean number of distinct compiler bugs
  detected.  As a
  result of our approach, which is available as an easy-to-use tool
  based on the widely used AFL fuzzer
  (\url{https://github.com/google/AFL}), we have reported more than
  100 previously undiscovered 
  bugs, subsequently fixed, bugs in important real-world compilers, and received a bug bounty for
  our efforts.  In the longest-running campaign, that targeting the
  {\tt solc} compiler for Solidity code, we were the first to
  report a large number of serious bugs, despite extensive
  fuzzing using AFL performed by the compiler developers
  and external contributors, over the same time frame, including
  OSS-fuzz continuous fuzzing.

\section{Mutation-Testing-Based Fuzzing}



\subsection{Mutation-Based Fuzzing}

One use of the term ``mutation'' appears in the context of the ``no-fuss'' \emph{mutation-based} fuzzing discussed above~\cite{ArtFuzz}.  A fuzzer such as AFL operates by executing the program under test (here, the compiler) on inputs (initially those in a corpus of example programs), using instrumentation to determine code coverage for each executed input.  The fuzzer then takes inputs that look interesting (e.g., uniquely cover some compiler code) and adds them to a \emph{queue}.  The basic loop repeatedly takes some input from the queue, \emph{mutates} it by making some essentially random change (e.g., flipping a single bit, or removing a random chunk of bytes), then executes the new, mutated input under instrumentation, and adds the new input to the queue if it satisfies the fuzzer's coverage-based criteria for interesting inputs.  The details of selecting inputs from the queue and determining how to mutate an input vary widely, and improving the effectiveness of this basic approach has been a major topic of recent software testing and security research~\cite{evalfuzz,BoehmeCR21,ArtFuzz}.  However, the inner fuzzing loop strategy is simple:

\begin{enumerate}
\item Select an input from the queue.
\item Mutate that input in order to obtain a new input.
\item Execute the new input, and if it is interesting, add it to the queue. Then repeat the process from step 1.
\end{enumerate}

Inputs that crash the compiler in step 3 are reported as bugs.  Using such a fuzzer is often extremely easy, requiring only 1) building the compiler with special instrumentation\footnote{With AFL this is fairly trivial, by using a drop-in compiler replacement, for C, C++, Rust; there are AFL variants for Go, Python, and other languages, as well.  AFL can also use QEMU to fuzz arbitrary binaries.  However, for compilers, it is usually best if possible to rebuild the compiler, since QEMU-based execution is much slower, and fuzz throughput is important.} and 2) finding a set of initial programs to use as a corpus.  


Our work focuses on improving step 2 of this process, in a way that is agnostic to how the details of the other aspects of fuzzing are implemented.  In particular, the problem with most approaches to mutation in the literature, for compiler fuzzing, is that changes such as byte-level-transformations almost always take compiling programs that exercise interesting compiler behavior, and transform them into programs that don't make it past early stages of parsing.  Alternative approaches to what are called ``havoc''-style mutations tend to involve solving constraints~\cite{Eclipser} or following taint~\cite{Angora}, which in the case of compilers tends to be ineffective, since the relationships are too complex to solve/follow.  A second common approach, providing a \emph{dictionary} of meaningful byte sequences in a language, is both burdensome on compiler developers and limited in effectiveness: a dictionary cannot, for example, help the fuzzer delete sub-units of code such as statements or blocks.

We propose a novel way to produce a much larger number of useful, interesting mutations for source code, without paying an analytical price that makes fuzzing practically infeasible for compilers, and without requiring \emph{any} additional effort on the part of compiler developers.

\subsection{Mutation Testing}

A different use of the term ``mutation'' appears in the field of mutation testing.  Mutation testing~\cite{MutationSurvey,budd1979mutation,demillo1978hints} is an approach to evaluating and improving software tests that works by introducing small syntactic changes into a program, under the assumption that if the original program was correct, then a program with slightly different semantics will be incorrect, and should be flagged as such by the tests.  Mutation testing is used in software testing research, occasionally in industry at-scale, and in some critical  open-source work~\cite{mutKernel,mutGoogle,mutFacebook}.

A mutation testing approach is defined by a set of mutation operators.  Such operators vary widely in the literature, though a few, such as deleting a small portion of code (such as a statement) or replacing arithmetic and relational operations (e.g., changing {\tt +} to {\tt -} or {\tt ==} to {\tt <=}), are very widely used.  Most mutation testing tools parse the code to be mutated do not work on code that does not parse.  However, recently there has been interest in performing mutation using purely syntactic operations, defined by a set of regular expressions~\cite{regexpMut}, implemented in a tool, the Universal Mutator (\url{https://github.com/agroce/universalmutator/}), with research and industry adoption.  Rather than taking a program, per se, this approach takes any code-like text and produces variants of the text corresponding to mutations.  The essence of this approach to mutation testing, which can be applied to ``any language,'' is essentially a transformation from arbitrary bytes to arbitrary bytes that, \emph{if the original bytes are ``code-like'' will tend to preserve that property}.  The resemblance to the mutations performed by mutation-based fuzzers is the core insight behind our approach.

\subsection{Combining Both Forms of Mutation}

The core of our approach is simply to add a new set of mutations to the repertoire of a mutation-based fuzzer.  These mutations are either traditional mutation operators or inspired by traditional mutation operators, but with changes made to satisfy the needs of fuzzing.  Unlike most changes made by mutation-based fuzzers, these mutations are likely to preserve the property that an input will get through a parser or trigger interesting optimizations.  The tendency to preserve such properties is natural, since the basis of mutation testing is to take an existing program and produce a set of new, similar programs.  If most mutation operators tended to produce uninteresting code that doesn't even compile, mutation testing would not be of much use.  Moreover, because our approach is based on the Universal Mutator tool~\cite{regexpMut}, the mutation operators used are essentially language-agnostic, and useful for fuzzing any syntactically ``conventional'' programming language (under which we include not only C-like languages, but even LISP-like languages). Staying within the theme of preserving code structure, we further enable an augmentation of our approach where we decompose existing test programs into constituent fragments that are then used to synthesize and mutate new inputs at runtime.

\subsection{Limitations}

The most important limitation for the mutation-testing-based approach is that if compiler \emph{crashes} are mostly uninteresting, fuzzing of this kind will probably not be very useful.  This applies, of course, to all AFL-style fuzzing, not just our approach.  For example, code that crashes a C or C++ compiler, but that includes extensive undefined behavior may well be ignored by developers.  Csmith~\cite{csmith} devotes a great deal of effort to avoiding generating such code.  On the other hand, many languages more recent than C and C++ attempt to provide a more ``total'' language where, while a program may be considered absurd by a human, fewer (or no) programs are undefined.  For example, smart contract languages such as those studied in this paper generally aim to make all programs that compile well-defined, or at least minimize the problem to more manageable cases such as order of evaluation of sub-expressions.  Similarly, Rust code without use of {\tt unsafe} should not crash the compiler, and any such crashes indicate possible bugs in the Rust compiler or type system.  

\section{Conclusions}

Using automated methods to find bugs is an important part of modern
compiler development.  For a variety of reasons, the use of
off-the-shelf fuzzers, especially AFL, is the most widely used such
approach.  Mutation-based fuzzers, however, were originally used to
test binary formats, and their heritage limits their effectiveness for
fuzzing compilers.  Most solutions to this problem either place significant
burden on compiler developers, or are ineffective, or both.  We show
that using ideas from mutation testing it is possible to significantly
improve AFL-based fuzzing of compilers without forcing developers to
provide grammars or a dictionary.  One of our two configurations
performed best for all compilers we investigated, in experiments,
sometimes dramatically so (finding more than twice as many bugs on
average as the best version of un-augmented AFL).  Using our approach
we reported more than 100 previously unknown bugs, that have been
fixed as a result, in important
compiler projects.

{\small {\bf {Acknowledgements:}}  A portion of this work was supported by the National Science Foundation under CCF-2129446.}

\balance

\bibliographystyle{ACM-Reference-Format}
\bibliography{bibliography}

\balance

\end{document}
